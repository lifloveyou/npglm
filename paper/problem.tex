\section{Problem Formulation}\label{sec:problem}
In this section, we introduce some important concepts and definitions used throughout the paper and formally define the problem of continuous-time relationship prediction.

\subsection{Heterogeneous Information Networks}

An information network is \emph{heterogeneous} if it contains multiple kinds of nodes and links. Formally, it is defined as a directed graph $G=(V,E)$ where $V = \cup_i V_i$ is the set of nodes composed of the union of all the node sets $V_i$ of type $i$. Similarly, $E=\cup_j E_j$ is the set of links constituted by the union of all the link sets $E_j$ of type $j$. Now we bring the definition of the \emph{network schema} \cite{sun2011pathsim} which is used to describe a heterogeneous information network in a meta-level:

\begin{definition}{(Network Schema)}
	The schema of a heterogeneous network $G$ is a graph $S_G=(\nu, \varepsilon)$ where $\nu$ is the set of different node types and $\varepsilon$ is the set of different link types in $G$.
\end{definition}

\begin{figure}
	\centering
	\scriptsize
	\begin{tikzpicture}[->,>=stealth',shorten >=1pt,auto,node distance=1.5cm,semithick]
	
	\node[state] (P)                    {$Paper$};
	\node[state] (V) [above=0.7cm of P] {$Venue$};
	\node[state] (T) [left=1.2cm  of P] {$Term$};
	\node[state] (A) [right=1.2cm of P] {$Author$};
	
	\path 
	(A) edge              node {write}     (P)
	(V) edge              node {publish}   (P)            
	(P) edge              node {mention}   (T)
	(P) edge [loop below] node {cite}      (P);
	
	\end{tikzpicture}
	\caption{Schema for DBLP bibliographic citation network.}
	\label{fig:schema}
\end{figure}

In this paper, we focus on the bibliographic network, in which \emph{Author}, \emph{Paper}, \emph{Venue}, and \emph{Term} are different node types denoted as $A$, $P$, $V$, and $T$, respectively. There are also \emph{write}, \emph{publish}, \emph{mention}, and \emph{cite} as different link types. The schema of this network has been depicted in Fig.~\ref{fig:schema}. 

As we mentioned in the Introduction section about heterogeneous networks, the concept of a link can be generalized to a relationship. In this case, a relationship could be either a single link, or a composite relation constituted by concatenation of multiple links that together has a particular semantic meaning. For example, the co-authorship relation in a bibliographic network with the schema shown in Fig.~\ref{fig:schema}, can be defined as the combination of two Author-write-Paper links, making Author-write-Paper-write-Author relation. When dealing with the link or relationship prediction in heterogeneous networks, we must exactly specify what kind of link or relationship we are going to predict. This is called the \emph{Target Relation} \cite{sun2012will}. In this paper, we aim to predict if and when an author in a bibliographic network will cite a paper from another author. Thus the target relation in this case would be Author-write-Paper-cite-Paper-write-Author.

\subsection{Dynamic Information Networks}
An information network is \emph{dynamic} when its nodes and linkage structure could change over time. That is, in a dynamic information network, all nodes and links are associated with a birth and death time, which denotes when they exist in the network. More formally, a dynamic network between time $t_1$ and $t_2$ is defined as $G^{t_1,t_2}=(V^{t_1,t_2}, E^{t_1,t_2})$ where $V^{t_1,t_2}$ and $E^{t_1,t_2}$ are the set of nodes and the set of links between time $t_1$ and $t_2$, respectively, which are defined as follows:
\[V^{t_1,t_2}=\{v: t_1< t_b(v)\le t_2<t_d(v)\}\]
\[E^{t_1,t_2}=\{e: t_1< t_b(e)\le t_2<t_d(e)\}\]
where $t_b(x)$ and $t_d(x)$ denotes the birth and death time of a network entity $x$ (which can be node or link,) respectively.

For the sake of simplicity, in this paper we assume that each network entity have only a single birth and death time, so we do not consider the cases of reoccurring or multiple links. Furthermore, without loss of generality, we assume that the node objects are persistent over time, and only linkage structure would change. This is due to the fact that an isolated node which does not participate in any link will not have any effect on the network and can be simply ignored. By this definition, a node will bear when it make a link with another node for the first time, and it dies when it loses all of its associated links. This can be formally stated as below:
\[\forall v\in V^{0,\infty}: t_b(v)=\min_e t_b(e), \ v\in e\in E^{0,\infty}\]
\[\forall v\in V^{0,\infty}: t_d(v)=\max_e t_d(e), \ v\in e\in E^{0,\infty}\]

In this paper, we consider the case that an information network is both dynamic and heterogeneous. This means that all network entities are associated with a type, and they also have their birth and death times, regardless of their type. The bibliographic network is an example of both dynamic and heterogeneous network. Whenever a new paper is published, a new Paper node will be added to network, alongside with some new Author, Term, and Venue nodes. Some linkages might also form between the existing nodes and the new ones. For example, if one of the authors of the new paper had written another paper before, a link would have been formed between her and her new paper.

\subsection{Continuous-Time Relationship Prediction}
Suppose that we are given a dynamic and heterogeneous information network as $G^{t_1,t_2}$, observed between the times $t_1$ and $t_2$, and is associated with a network schema $S_G$. Now, given the target relation $R$, the aim of continuous-time relationship prediction is to continuously forecast the building time $t\ge t_2$ of $R$ for a given pair of nodes in $G^{t_1,t_2}$.

To solve this problem, we first introduce a feature extraction framework to cope with both the dynamicity and heterogeneity of the network, and then propose a non-parametric model which utilizes the extracted features to perform predictions about the building time of the target relationship.
