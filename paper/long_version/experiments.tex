\section{Experiments on Real-World Data}\label{sec:results}

We apply \npglm with the proposed feature set on a number of real-world datasets to evaluate its effectiveness and compare its performance in predicting the relationship building time vis-\`a-vis state-of-the-art models. 

\subsection{Dataset}
\subsubsection{DBLP}
We use DBLP bibliographic citation network, provided by \cite{tang2008aminer}, which has both attributes of dynamicity and heterogeneity. The network contains four types of objects: authors, papers, venues, and terms. The network schema of this dataset is depicted in Fig~\ref{fig:schema:dblp}. Based on the publication venue of the papers, we limited the original DBLP dataset to those papers that are published in venues relative to the theoretical computer science. This will result in about 37k papers ranging from 1969 to 2016, published in 38 venues.

%The demographic statistics of both networks is presented in Table~\ref{table:dataset}.

\subsubsection{Delicious}
Another dynamic and heterogeneous dataset we use in our experiments is the Delicious bookmarking dataset from \cite{Cantador:RecSys2011}, with a network schema presented in Fig~\ref{fig:schema:delicious}. It contains three types of objects, namely users, bookmarks, and tags. The dataset includes bookmarking timestamps from May 2006 to October 2010. 

\subsubsection{MovieLens}
The third heterogeneous dataset with dynamic characteristics has been extracted from MovieLens personalized movie recommendations website, provided by \cite{harper2015}. The dataset comprises seven types of objects, that are users, movies, tags, genres, actors, directors, and countries, as illustrated by the network schema in the Fig~\ref{fig:schema:movielens}. It contains user-movie rating timestamps ranging from September 1997 to January 2009.

\begin{table}[t]
	\centering
	\caption{Demographic Statistics of Real-World Datasets}
	\label{table:dataset}
	\scriptsize
	\begin{tabu} to \columnwidth {l l X[l] X[l] X[r]}
		\toprule
		Dataset & Time Span & Entity & Title & Count\\
		\midrule % In-table horizontal line
		\multirow{8}{*}{DBLP} & \multirow{8}{2cm}{From 1969 to 2016}
		& \multirow{4}{*}{Nodes}
		& $Author$ & 15,929 \\ % Content row 1
		& & & $Paper$ & 37,077  \\ % Content row 2
		& & & $Venue$ & 38 \\ % Content row 3
		& & & $Term$ & 12,028 \\ % Content row 3
		\cmidrule{3-5}
		& & \multirow{4}{*}{Links}
		& write & 100,797 \\ % Content row 1
		& & & cite & 165,904 \\ % Content row 2
		& & & publish & 42,872 \\ % Content row 3
		& & & mention & 284,156 \\ % Content row 3
		
		\midrule % In-table horizontal line
		\multirow{6}{*}{Delicious} & \multirow{6}{2cm}{From May 2006 to Oct 2010}
		& \multirow{3}{*}{Nodes}
		& $User$ & 1,714 \\ % Content row 1
		& & & $Tag$ & 21,956  \\ % Content row 2
		& & & $Bookmark$ & 30,998 \\ % Content row 3
		\cmidrule{3-5}
		& & \multirow{3}{*}{Links}
		& contact & 15,329 \\ % Content row 1
		& & & post & 437,594 \\ % Content row 2
		& & & has-tag & 437,594 \\ % Content row 3
		
		\midrule % In-table horizontal line
		\multirow{13}{*}{MovieLens} & \multirow{13}{2cm}{From Sep 1997 to Jan 2009}
		& \multirow{7}{*}{Nodes}
		& $User$ & 1,421 \\ % Content row 1
		& & & $Movie$ & 5,660  \\ % Content row 2
		& & & $Actor$ & 6,176 \\ % Content row 3
		& & & $Director$ & 2,401 \\ % Content row 3
		& & & $Genre$ & 19 \\ % Content row 3
		& & & $Tag$ & 5,561 \\ % Content row 3
		& & & $Country$ & 63 \\ % Content row 3
		\cmidrule{3-5}
		& & \multirow{6}{*}{Links}
		& rate & 855,599 \\ % Content row 1
		& & & play-in & 231,743 \\ % Content row 2
		& & & direct & 10,156 \\ % Content row 3
		& & & has-genre & 20,810 \\ % Content row 3
		& & & has-tag & 47,958 \\ % Content row 3
		& & & produced-in & 10,198 \\ % Content row 3
		\bottomrule % Bottom horizontal line
	\end{tabu}
\end{table}

\subsection{Experiment Settings}
\subsubsection{Comparison Methods}
To challenge the performance of \npglm, we use the state of the art generalized linear model-based framework proposed in \cite{sun2012will} with Exponential, Rayleigh, and Weibull as distributions with different shapes, denoted as \textsc{Exp-Glm}, \textsc{Ray-Glm}, and \textsc{Wbl-Glm}, respectively. To examine the effect of considering the dynamicity of the network on the performance of the models, we evaluated each one with two different feature sets: \emph{dynamic} and \emph{static}. Dynamic feature set is our proposed features which captures the dynamicity of the network. On the contrary, static feature set only uses the very last snapshot of the network just before the beginning of the observation window, failing to reflect the temporal dynamics of the network.
\subsubsection{Performance Measures}
We assess different methods using a number of evaluation metrics which are described in the following:
\begin{itemize}
\item Mean Absolute Error (MAE): this measures the expected absolute error between the predicted time values and the ground truth, and is computed as:
\[MAE(\mb{t},\hat{\mb{t}}) = \frac{1}{N}\sum_{i=1}^{N}\left|t_i-\hat{t}_i\right|\]
\item Mean Relative Error (MRE): this measures the expected relative absolute error between the predicted time values and the ground truth, and is computed as:
\[MRE(\mb{t},\hat{\mb{t}}) = \frac{1}{N}\sum_{i=1}^{N}\left|\frac{t_i-\hat{t}_i}{t_i}\right|\]
\item Root Mean Squared Error (RMSE): this measures the root of the expected squared error between the predicted time values and the ground truth, and is computed as:
\[RMSE(\mb{t},\hat{\mb{t}}) = \sqrt{\frac{1}{N}\sum_{i=1}^{N}\left(t_i-\hat{t}_i\right)^2}\]
\item Mean Squared Logarithmic Error (MSLE): this measures the expected value of the squared logarithmic error between the predicted time values and the ground truth, and is computed as:
\[RMSE(\mb{t},\hat{\mb{t}}) = \frac{1}{N}\sum_{i=1}^{N}\left(\log{(1+t_i)}-log{(1+\hat{t}_i)}\right)^2\]
\item Median Absolute Error (MDAE): this measures the median of the absolute errors between the predicted time values and the ground truth, and is computed as:
\[MDAE(\mb{t},\hat{\mb{t}}) = median(\left|t_1-\hat{t}_1\right|\dots\left|t_N-\hat{t}_N\right|)\]
\item Concordance Index (CI): this metric is one of the most widely used performance measures for survival models that estimates how good the model performs at ranking predicted times \cite{harrell1982evaluating}. It can be seen as the fraction of all pairs of samples whose predicted times are correctly ordered among all samples that can be ordered, and is considered as the generalization of the Area Under Receiver Operating Characteristic Curve (AUC) when we are dealing with censored data \cite{steck2008ranking}.
\end{itemize}

%We use 5-fold cross-validation and report the average results for all the experiments in this section. 
\subsubsection{Experiment Setup}
We confined the dataset to the papers which has been published between the years 1995 and 2016, and selected the authors who had published more than 5 papers in the Feature Extraction Window of each experiment. Following the method described for feature extraction in Section~\ref{sec:features}, we extracted 19 dynamic meta-path-based features. The rate parameter of features were determines using a separate validation set to get the best results. As \npglm needs the data samples sorted by their corresponding time variables, the samples were sorted according to their recorded time. Finally, since the features extracted for different meta-paths have different scales, we normalized the data using Z-Score. In all experiments, the author citation relation ($A\rightarrow P\rightarrow P\leftarrow A$) were chosen as the target relation.



\descr{Experiment Results.}
In the first set of experiments, we evaluated the prediction power of different models using different feature sets, and assessed how well they generalize in different configurations for the feature extraction window and the observation window. Under each configuration, we considered the median of the distribution $f_T(t\mid \mb{x}_{test})$ as the predicted time for that sample and then compared it to the ground truth time $t_{test}$. \emph{Mean Absolute Error} (MAE) and \emph{Mean Relative Error} (MRE) are used to measure the accuracy of the predicted values.  

In Table~\ref{table:results}, we set $\Phi=10$ and varied $\Omega$ within the set $\{3,6,9\}$. The MAE and MRE of all models under both dblp-db and dblp-th networks, and using both dynamic and static feature sets has been recored. We see that in both networks, \npglm using the dynamic features is superior to the other models. For instance, when $\Omega=3$, our model \npglm can obtain an MAE of 0.21 for the dblp-db dataset, which is 28.5\% lower than the MAE obtained by its closest competitor, \textsc{Exp-Glm}. As of MRE, \npglm achieves 0.16 on dblp-db, which is 25\% lower than \textsc{Exp-Glm}. 
On dblp-th, \npglm reduces the MAE and MRE by 36\% and 11\%, respectively, relative to \textsc{Exp-Glm}. Comparable results hold for other values of $\Omega$ as well. Moreover, in this table it is evident that using the dynamic features has a positive impact on the performance of all models and decreases their prediction error.

Table~\ref{table:results2} compares the MAE and MRE of different models when we set $\Omega=6$ and selected $\Phi$ from the set $\{5,10,15\}$. Again the results of all models have been obtained under both dblp-db and dblp-th networks, and using both static and dynamic feature sets. It is clear that \npglm has performed better than other baselines, even when it uses static features.
When $\Phi=15$ for example, \npglm achieved an MAE of 0.62 and an MRE of 0.35 on the dblp-db dataset, which are respectively 22\% and 14\% lower than \textsc{Exp-Glm}'s. On dblp-th, \npglm caused a 7\% cut in MAE and a 30\% cut in MRE compared to \textsc{Exp-Glm}'s results.
Here, the obtained results also show that leveraging more historical data to extract features could lead to more accurate predictions and lower error rates.

\begin{figure}[t]
	\hfill
	\subfloat[Rayleigh distribution]{
		\begin{tikzpicture}[trim axis left, trim axis right]
		\begin{axis}
		[
		tiny,
		width=0.56\columnwidth,
		height=4.5cm,
		legend pos=north east,
		legend style={font=\scriptsize,nodes={scale=0.75, transform shape}},
		grid,
		y tick label style={
			/pgf/number format/.cd,
			fixed,
			fixed zerofill,
			precision=2,
			/tikz/.cd
		},
		xlabel=$N_c$,
		%xticklabel style={rotate=90},
		ylabel=MAE,
		ylabel shift = -4 pt,
		ymax=0.2,
		ymin=0.06,
		%xmin=0,
		%xmax=2100,
		ytick={0.08,0.10,...,0.2},
		xtick={0,40,...,200},
		legend entries={$N_o=200$, $N_o=300$, $N_o=400$},
		]
		\addplot[color=cyan,mark=*,mark size=1.1,thick] table{results/mae_ray_200.txt};
		\addplot[color=orange,mark=triangle*,mark size=1.5,thick] table{results/mae_ray_300.txt};
		\addplot[color=purple,mark=square*,mark size=1.1,thick] table{results/mae_ray_400.txt};
		\end{axis}
		\end{tikzpicture}
	}\hspace{1cm}    
	\subfloat[Gompertz distribution]{
		\begin{tikzpicture}[trim axis left, trim axis right]
		\begin{axis}
		[
		tiny,
		width=0.56\columnwidth,
		height=4.5cm,
		legend pos=north east,
		legend style={font=\scriptsize,nodes={scale=0.75, transform shape}},
		grid,
		y tick label style={
			/pgf/number format/.cd,
			fixed,
			fixed zerofill,
			precision=2,
			/tikz/.cd
		},
		xlabel=$N_c$,
		%xticklabel style={rotate=90},
		ylabel=MAE,
		ylabel shift = -4 pt,
		ymax=0.24,
		ymin=0.03,
		%xmin=0,
		%xmax=2100,
		ytick={0.06,0.09,...,0.21},
		xtick={0,40,...,200},
		legend entries={$N_o=200$, $N_o=300$, $N_o=400$},
		]
		\addplot[color=cyan,mark=*,mark size=1.1,thick] table{results/mae_gom_200.txt};
		\addplot[color=orange,mark=triangle*,mark size=1.5,thick] table{results/mae_gom_300.txt};
		\addplot[color=purple,mark=square*,mark size=1.1,thick] table{results/mae_gom_400.txt};
		\end{axis}
		\end{tikzpicture}
	}
	\caption{\npglm's mean absolute error (MAE) vs the number of censored samples ($N_c$) for different number of observed samples ($N_o$).}
	\label{fig:syn-mae-c}
\end{figure}

\begin{figure*}[t]
\centering
%\hfill
	\subfloat[DBLP]{
		\begin{tikzpicture}[trim axis left, trim axis right]
		\begin{axis}
		[
		tiny,
		width=0.56\columnwidth,
		height=4.5cm,
		legend pos=north west,
		legend style={font=\tiny,nodes={scale=0.75, transform shape}},
		grid,
		y tick label style={
			/pgf/number format/.cd,
			fixed,
			fixed zerofill,
			precision=1,
			/tikz/.cd
		},
		xlabel=Absolute Error,
		ylabel=Prediction Accuracy,
		ylabel shift = -4 pt,
		%xticklabel style={rotate=90},
		ymax=1.1,
		xmin=0,
		xmax=3.5,
		ytick={0.1,0.2,...,0.9,1.0},
		xtick={0.5,1.0,...,3},
%		restrict x to domain=0:900,
		legend entries={NP-GLM, WBL-GLM, EXP-GLM, RAY-GLM},
		]
		\addplot[color=purple,mark=square*,mark size=1.1,thick] table{results/db_np.txt};
		\addplot[color=cyan,mark=triangle*,mark size=1.1,thick] table{results/db_wbl.txt};
		\addplot[color=orange,mark=*,mark size=1.5,thick] table{results/db_exp.txt};
		\addplot[color=green,mark=diamond*,mark size=1.5,thick] table{results/db_ray.txt};
		\end{axis}
		\end{tikzpicture}
	}
\hfil
	\subfloat[Delicious]{
	\begin{tikzpicture}[trim axis left, trim axis right]
	\begin{axis}
	[
	tiny,
	width=0.56\columnwidth,
	height=4.5cm,
	legend pos=north west,
	legend style={font=\tiny,nodes={scale=0.75, transform shape}},
	grid,
	y tick label style={
		/pgf/number format/.cd,
		fixed,
		fixed zerofill,
		precision=1,
		/tikz/.cd
	},
	xlabel=Absolute Error,
	ylabel=Prediction Accuracy,
	ylabel shift = -4 pt,
	%xticklabel style={rotate=90},
	ymax=1.1,
	xmin=0,
	xmax=3.5,
	ytick={0.1,0.2,...,0.9,1.0},
	xtick={0.5,1.0,...,3},
	%		restrict x to domain=0:900,
	legend entries={NP-GLM, WBL-GLM, EXP-GLM, RAY-GLM},
	]
	\addplot[color=purple,mark=square*,mark size=1.1,thick] table{results/dl_np.txt};
	\addplot[color=cyan,mark=triangle*,mark size=1.1,thick] table{results/dl_wbl.txt};
	\addplot[color=orange,mark=*,mark size=1.5,thick] table{results/dl_exp.txt};
	\addplot[color=green,mark=diamond*,mark size=1.5,thick] table{results/dl_ray.txt};
	\end{axis}
	\end{tikzpicture}
	}
\hfil
	\subfloat[MovieLens]{
		\begin{tikzpicture}[trim axis left, trim axis right]
		\begin{axis}
		[
		tiny,
		width=0.56\columnwidth,
		height=4.5cm,
		legend pos=north west,
		legend style={font=\tiny,nodes={scale=0.75, transform shape}},
		grid,
		y tick label style={
			/pgf/number format/.cd,
			fixed,
			fixed zerofill,
			precision=1,
			/tikz/.cd
		},
		xlabel=Absolute Error,
		ylabel=Prediction Accuracy,
		ylabel shift = -4 pt,
		%xticklabel style={rotate=90},
		ymax=1.1,
		xmin=0,
		xmax=3.5,
		ytick={0.1,0.2,...,0.9,1.0},
		xtick={0.5,1.0,...,3},
		%		restrict x to domain=0:900,
		legend entries={NP-GLM, WBL-GLM, EXP-GLM, RAY-GLM},
		]
		\addplot[color=purple,mark=square*,mark size=1.1,thick] table{results/mv_np.txt};
		\addplot[color=cyan,mark=triangle*,mark size=1.1,thick] table{results/mv_wbl.txt};
		\addplot[color=orange,mark=*,mark size=1.5,thick] table{results/mv_exp.txt};
		\addplot[color=green,mark=diamond*,mark size=1.5,thick] table{results/mv_ray.txt};
		\end{axis}
		\end{tikzpicture}
	}
	\caption{Prediction accuracy of different methods vs the maximum tolerated absolute error on different datasets.}
	\label{fig:real}
\end{figure*}

In the final experiment, we investigated for what fraction of test samples a model can have a lower prediction error than a threshold. In other words, to evaluate the prediction accuracy of a model, we record the fraction of test samples for which the difference between their true times and the predicted ones are lower than a given threshold, called \emph{tolerated error}. To this end, we again used the median as the predicted value by all models that were trained using our dynamic features. The feature extraction and observation windows were set to 10 and 6, respectively, and the results for different tolerated errors in range $\{0.5, 1.0, \dots, 3.0\}$ were plotted in Fig~\ref{fig:real}. The result demonstrates that \npglm can achieve a higher accuracy in all cases relative to other baselines, especially when dealing with lower tolerated errors.

%\begin{table*}[t]
%	\centering
%	\caption{Performance Comparison of Different Methods Under Different Lengths for Observation Window when $\Phi=10$}
%	\label{table:results}
%	\scriptsize
%	\begin{tabu} to \textwidth {X[l] c l X[c] X[c] c X[c] X[c] c X[c] X[c]}
%		\toprule
%		& & 
%		& \multicolumn{2}{c}{$\Omega=3$} & & \multicolumn{2}{c}{$\Omega=6$} & & \multicolumn{2}{c}{$\Omega=9$}\\
%		\cmidrule(l){4-5} \cmidrule{7-8} \cmidrule{10-11}
%		%\cmidrule(l){2-3} \cmidrule{5-7}
%		Dataset & Feature &
%		Model & MAE & MRE & & MAE & MRE & & MAE & MRE\\
%		\midrule
%		\multirow{8}{*}{dblp-db}
%		& \multirow{4}{*}{\rotatebox{90}{Dynamic}}
%		& \npglm & $\bm{0.21\pm0.02}$ & $\bm{0.16\pm0.07}$ & & $\bm{0.76\pm0.05}$ & $\bm{0.34\pm0.03}$ & & $\bm{1.10\pm0.09}$ & $\bm{0.43\pm0.02}$ \\
%		& & \textsc{Exp-Glm} & $0.27\pm0.09$ & $0.20\pm0.02$ & & $0.83\pm0.09$ & $0.43\pm0.06$ & & $1.39\pm0.04$ & $0.66\pm0.01$ \\
%		& & \textsc{Ray-Glm} & $0.40\pm0.04$ & $0.32\pm0.05$ & & $1.90\pm0.06$ & $1.00\pm0.09$ & & $2.56\pm0.01$ & $1.60\pm0.02$ \\
%		& & \textsc{Gom-Glm} & $0.38\pm0.09$ & $0.32\pm0.09$ & & $1.89\pm0.07$ & $1.23\pm0.02$ & & $3.30\pm0.07$ & $2.03\pm0.01$ \\
%		
%		\cmidrule{2-11}
%		& \multirow{4}{*}{\rotatebox{90}{Static}}
%		& \npglm & $0.27\pm0.01$ & $0.20\pm0.01$ & & $0.85\pm0.03$ & $0.45\pm0.02$ & & $1.32\pm0.07$ & $0.56\pm0.02$ \\
%		& & \textsc{Exp-Glm} & $0.33\pm0.01$ & $0.24\pm0.01$ & & $0.93\pm0.03$ & $0.51\pm0.02$ & & $1.54\pm0.06$ & $0.74\pm0.05$ \\
%		& & \textsc{Ray-Glm} & $0.48\pm0.02$ & $0.37\pm0.02$ & & $2.16\pm0.06$ & $1.30\pm0.04$ & & $3.37\pm0.12$ & $1.84\pm0.11$ \\
%		& & \textsc{Gom-Glm} & $0.48\pm0.01$ & $0.36\pm0.01$ & & $2.28\pm0.02$ & $1.37\pm0.03$ & & $4.24\pm0.10$ & $2.28\pm0.06$ \\
%		
%		\midrule
%		\multirow{8}{*}{dblp-th}
%		& \multirow{4}{*}{\rotatebox{90}{Dynamic}}
%		& \npglm & $\bm{0.22\pm0.03}$ & $\bm{0.17\pm0.08}$ & & $\bm{0.75\pm0.09}$ & $\bm{0.36\pm0.02}$ & & $\bm{1.06\pm0.09}$ & $\bm{0.47\pm0.08}$ \\
%		& & \textsc{Exp-Glm} & $0.30\pm0.03$ & $0.19\pm0.07$ & & $0.75\pm0.09$ & $0.43\pm0.06$ & & $1.37\pm0.03$ & $0.60\pm0.02$ \\
%		& & \textsc{Ray-Glm} & $0.41\pm0.01$ & $0.34\pm0.01$ & & $1.83\pm0.06$ & $1.11\pm0.02$ & & $3.01\pm0.07$ & $1.56\pm0.04$ \\
%		& & \textsc{Gom-Glm} & $0.38\pm0.06$ & $0.30\pm0.09$ & & $1.88\pm0.05$ & $1.23\pm0.02$ & & $3.97\pm0.06$ & $2.14\pm0.02$ \\
%		
%		\cmidrule{2-11}
%		& \multirow{4}{*}{\rotatebox{90}{Static}}
%		& \npglm & $0.28\pm0.01$ & $0.21\pm0.01$ & & $0.89\pm0.04$ & $0.47\pm0.02$ & & $1.37\pm0.07$ & $0.57\pm0.03$ \\
%		& & \textsc{Exp-Glm} & $0.34\pm0.02$ & $0.25\pm0.02$ & & $0.98\pm0.03$ & $0.54\pm0.02$ & & $1.59\pm0.07$ & $0.77\pm0.04$ \\
%		& & \textsc{Ray-Glm} & $0.50\pm0.03$ & $0.39\pm0.02$ & & $2.26\pm0.07$ & $1.36\pm0.05$ & & $3.54\pm0.13$ & $1.92\pm0.08$ \\
%		& & \textsc{Gom-Glm} & $0.50\pm0.01$ & $0.38\pm0.02$ & & $2.38\pm0.06$ & $1.43\pm0.07$ & & $4.41\pm0.14$ & $2.40\pm0.07$ \\
%		
%		\bottomrule
%	\end{tabu}
%\end{table*}

\begin{figure}[t]
	\centering
	\hfill
	\subfloat[Mean Absolute Error]{
		\begin{tikzpicture}[trim axis left, trim axis right]
		\begin{axis}
		[
		tiny,
		width=0.56\columnwidth,
		height=4.5cm,
		legend pos=north east,
		legend style={font=\tiny,nodes={scale=0.75, transform shape}},
		grid,
		y tick label style={
			/pgf/number format/.cd,
			fixed,
			fixed zerofill,
			precision=1,
			/tikz/.cd
		},
		xlabel=\# Snapshots,
		ylabel=MAE,
		ylabel shift = -4 pt,
		%xticklabel style={rotate=90},
%		ymax=1.1,
		xmin=0,
		xmax=21,
%		ytick={0.1,0.2,...,0.9,1.0},
		xtick={3,6,...,18},
		%		restrict x to domain=0:900,
		legend entries={NP-GLM, WBL-GLM},
		]
		\addplot[color=purple,mark=square*,mark size=1.1,thick] table{results/dl_snap_mae_np.txt};
		\addplot[color=cyan,mark=triangle*,mark size=1.1,thick] table{results/dl_snap_mae_wbl.txt};
		\end{axis}
		\end{tikzpicture}
	}
	\hfill
	\subfloat[Concordance Index]{
		\begin{tikzpicture}[trim axis left, trim axis right]
		\begin{axis}
		[
		tiny,
		width=0.56\columnwidth,
		height=4.5cm,
		legend pos=north west,
		legend style={font=\tiny,nodes={scale=0.75, transform shape}},
		grid,
		y tick label style={
			/pgf/number format/.cd,
			fixed,
			fixed zerofill,
			precision=1,
			/tikz/.cd
		},
		xlabel=\# Snapshots,
		ylabel=CI,
		ylabel shift = -4 pt,
		%xticklabel style={rotate=90},
		%		ymax=1.1,
		xmin=0,
%		xmax=3.5,
		%		ytick={0.1,0.2,...,0.9,1.0},
		xtick={3,6,...,18},
		%		restrict x to domain=0:900,
		legend entries={NP-GLM, WBL-GLM},
		]
		\addplot[color=purple,mark=square*,mark size=1.1,thick] table{results/dl_snap_ci_np.txt};
		\addplot[color=cyan,mark=triangle*,mark size=1.1,thick] table{results/dl_snap_ci_wbl.txt};
		\end{axis}
		\end{tikzpicture}
	}
	\caption{Effect of choosing different number of snapshots on performance of different methods using Delicious dataset.}
	\label{fig:delta:delicious}
\end{figure}


\begin{figure}[t]
	\centering
	\hfill
	\subfloat[Mean Absolute Error]{
		\begin{tikzpicture}[trim axis left, trim axis right]
		\begin{axis}
		[
		tiny,
		width=0.56\columnwidth,
		height=4.5cm,
		legend pos=north east,
		legend style={font=\tiny,nodes={scale=0.75, transform shape}},
		grid,
		y tick label style={
			/pgf/number format/.cd,
			fixed,
			fixed zerofill,
			precision=1,
			/tikz/.cd
		},
		xlabel=\# Snapshots,
		ylabel=MAE,
		ylabel shift = -4 pt,
		%xticklabel style={rotate=90},
%				ymax=18,ymin=10,
		xmin=0,
%		xmax=3.5,
		%		ytick={0.1,0.2,...,0.9,1.0},
		xtick={3,6,...,18},
		%		restrict x to domain=0:900,
		legend entries={NP-GLM, WBL-GLM},
		]
		\addplot[color=purple,mark=square*,mark size=1.1,thick] table{results/mv_snap_mae_np.txt};
		\addplot[color=cyan,mark=triangle*,mark size=1.1,thick] table{results/mv_snap_mae_wbl.txt};
		\end{axis}
		\end{tikzpicture}
	}
	\hfill
	\subfloat[Concordance Index]{
		\begin{tikzpicture}[trim axis left, trim axis right]
		\begin{axis}
		[
		tiny,
		width=0.56\columnwidth,
		height=4.5cm,
		legend pos=north west,
		legend style={font=\tiny,nodes={scale=0.75, transform shape}},
		grid,
		y tick label style={
			/pgf/number format/.cd,
			fixed,
			fixed zerofill,
			precision=1,
			/tikz/.cd
		},
		xlabel=\# Snapshots,
		ylabel=CI,
		ylabel shift = -4 pt,
		%xticklabel style={rotate=90},
				ymax=0.9,ymin=0.2,
		xmin=0,
%		xmax=3.5,
		%		ytick={0.1,0.2,...,0.9,1.0},
		xtick={3,6,...,18},
		%		restrict x to domain=0:900,
		legend entries={NP-GLM, WBL-GLM},
		]
		\addplot[color=purple,mark=square*,mark size=1.1,thick] table{results/mv_snap_ci_np.txt};
		\addplot[color=cyan,mark=triangle*,mark size=1.1,thick] table{results/mv_snap_ci_wbl.txt};
		\end{axis}
		\end{tikzpicture}
	}
	\caption{Effect of choosing different number of snapshots on performance of different methods using MovieLens dataset.}
	\label{fig:delta:MovieLens}
\end{figure}










\begin{figure}[t]
	\centering
	\hfill
	\subfloat[Mean Absolute Error]{
		\begin{tikzpicture}[trim axis left, trim axis right]
		\begin{axis}
		[
		tiny,
		width=0.56\columnwidth,
		height=4.5cm,
		legend pos=north east,
		legend style={font=\tiny,nodes={scale=0.75, transform shape}},
		grid,
		y tick label style={
			/pgf/number format/.cd,
			fixed,
			fixed zerofill,
			precision=1,
			/tikz/.cd
		},
		xlabel=$\Delta$,
		ylabel=MAE,
		ylabel shift = -4 pt,
		%xticklabel style={rotate=90},
		%		ymax=1.1,
		xmin=0,
		xmax=3.5,
		%		ytick={0.1,0.2,...,0.9,1.0},
		xtick={0.5,1.0,...,3},
		%		restrict x to domain=0:900,
		legend entries={NP-GLM, WBL-GLM},
		]
		\addplot[color=purple,mark=square*,mark size=1.1,thick] table{results/delta_mae_np.txt};
		\addplot[color=cyan,mark=triangle*,mark size=1.1,thick] table{results/delta_mae_wbl.txt};
		\end{axis}
		\end{tikzpicture}
	}
	\hfill
	\subfloat[Concordance Index]{
		\begin{tikzpicture}[trim axis left, trim axis right]
		\begin{axis}
		[
		tiny,
		width=0.56\columnwidth,
		height=4.5cm,
		legend pos=north west,
		legend style={font=\tiny,nodes={scale=0.75, transform shape}},
		grid,
		y tick label style={
			/pgf/number format/.cd,
			fixed,
			fixed zerofill,
			precision=1,
			/tikz/.cd
		},
		xlabel=$\Delta$,
		ylabel=CI,
		ylabel shift = -4 pt,
		%xticklabel style={rotate=90},
		%		ymax=1.1,
		xmin=0,
		xmax=3.5,
		%		ytick={0.1,0.2,...,0.9,1.0},
		xtick={0.5,1.0,...,3},
		%		restrict x to domain=0:900,
		legend entries={NP-GLM, WBL-GLM},
		]
		\addplot[color=purple,mark=square*,mark size=1.1,thick] table{results/delta_ci_np.txt};
		\addplot[color=cyan,mark=triangle*,mark size=1.1,thick] table{results/delta_ci_wbl.txt};
		\end{axis}
		\end{tikzpicture}
	}
	\caption{Effect of choosing different values for $\Delta$ on performance of different methods using Delicious dataset.}
	\label{fig:snaps:delicious}
\end{figure}


\begin{figure}[t]
	\centering
	\hfill
	\subfloat[Mean Absolute Error]{
		\begin{tikzpicture}[trim axis left, trim axis right]
		\begin{axis}
		[
		tiny,
		width=0.56\columnwidth,
		height=4.5cm,
		legend pos=north east,
		legend style={font=\tiny,nodes={scale=0.75, transform shape}},
		grid,
		y tick label style={
			/pgf/number format/.cd,
			fixed,
			fixed zerofill,
			precision=1,
			/tikz/.cd
		},
		xlabel=$\Delta$,
		ylabel=MAE,
		ylabel shift = -4 pt,
		%xticklabel style={rotate=90},
		ymax=18,ymin=10,
		xmin=0,
		xmax=3.5,
		%		ytick={0.1,0.2,...,0.9,1.0},
		xtick={0.5,1.0,...,3},
		%		restrict x to domain=0:900,
		legend entries={NP-GLM, WBL-GLM},
		]
		\addplot[color=purple,mark=square*,mark size=1.1,thick] table{results/mv_delta_mae_np.txt};
		\addplot[color=cyan,mark=triangle*,mark size=1.1,thick] table{results/mv_delta_mae_wbl.txt};
		\end{axis}
		\end{tikzpicture}
	}
	\hfill
	\subfloat[Concordance Index]{
		\begin{tikzpicture}[trim axis left, trim axis right]
		\begin{axis}
		[
		tiny,
		width=0.56\columnwidth,
		height=4.5cm,
		legend pos=north west,
		legend style={font=\tiny,nodes={scale=0.75, transform shape}},
		grid,
		y tick label style={
			/pgf/number format/.cd,
			fixed,
			fixed zerofill,
			precision=1,
			/tikz/.cd
		},
		xlabel=$\Delta$,
		ylabel=CI,
		ylabel shift = -4 pt,
		%xticklabel style={rotate=90},
		ymax=0.9,ymin=0.2,
		xmin=0,
		xmax=3.5,
		%		ytick={0.1,0.2,...,0.9,1.0},
		xtick={0.5,1.0,...,3},
		%		restrict x to domain=0:900,
		legend entries={NP-GLM, WBL-GLM},
		]
		\addplot[color=purple,mark=square*,mark size=1.1,thick] table{results/mv_delta_ci_np.txt};
		\addplot[color=cyan,mark=triangle*,mark size=1.1,thick] table{results/mv_delta_ci_wbl.txt};
		\end{axis}
		\end{tikzpicture}
	}
	\caption{Effect of choosing different values for $\Delta$ on performance of different methods using MovieLens dataset.}
	\label{fig:snaps:MovieLens}
\end{figure}

