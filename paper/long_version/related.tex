\section{Related Works}\label{sec:related}
\newcommand{\etal}{\textit{et~al}.}

The problem of link prediction has been studied extensively in recent years and many approaches have been proposed to solve this problem \cite{wang2015link,wang2014review}.
Previous works on time-aware link prediction have mostly considered temporality in analyzing the long-term network trend over time \cite{dhote2013survey}. Authors in \cite{potgieter2009temporality} have shown that temporal metrics are an extremely valuable new contribution to link prediction, and should be used in future applications. 
%\cite{tylenda2009towards} incorporated temporal information available on evolving social networks for link prediction tasks and proposed a novel node-centric approach to the evaluation of link prediction. 
Dunlavy \etal{} focused on the problem of periodic temporal link prediction \cite{dunlavy2011temporal}. They focused on bipartite graphs that evolve over time and also considered weighted matrix that contained multilayer data and tensor-based methods for predicting future links.
Oyama \etal{} solved the problem of cross-temporal link prediction, in which the links among nodes in different time frames are inferred \cite{oyama2011cross}. They mapped data objects in different time frames into a common low-dimensional latent feature space, and identified the links on the basis of the distance between the data objects.
{\"O}zcan \etal{} proposed a novel link prediction method for evolving networks based on NARX neural network \cite{ozcan2016temporal}. They take the correlation between the quasi-local similarity measures and temporal evolutions of link occurrences information into account by using NARX for multivariate time series forecasting.
Yu \etal{} developed a novel temporal matrix factorization model to explicitly represent the network as a function of time \cite{yu2017temporally}. They provided results for link prediction as an specific example and showed that their model performs better than the state-of-the-art techniques.

The most relevant works to this study are available in \cite{hajibagheri2016leveraging, aggarwal2012dynamic, sett2017temporal, sun2012will}. The Authors in \cite{hajibagheri2016leveraging} approaches the problem of time series link prediction by extracting simple temporal features from the time series, such as mean, (weighted) moving average, and exponential smoothing besides some topological features like common neighbor and adamic-adar. But their method is designed for homogeneous networks and fail to consider the heterogeneity of the modern networks. Aggarwal \etal{} \cite{aggarwal2012dynamic} tackle the link prediction problem in both dynamic and heterogeneous information networks using a dynamic clustering approach alongside with content-based and structural models. However they aim to solve the conventional link prediction problem, not the continuous-time relationship prediction problem studied in this paper. In \cite{sett2017temporal}, the authors proposed a feature set, called TMLP, well suited for link prediction in dynamic and heterogeneous information networks. Although their proposed feature set cope with both dynamicity and heterogeneity of the network, it cannot be extended for the generalized problem of relationship prediction and is only designed for solving the simpler link prediction problem.

Most of the aforementioned works answered the question of \emph{whether} a link will appear in the network. To the best of our knowledge, the only work that has focused on the continuous-time relationship prediction problem is proposed by Sun \etal{} \cite{sun2012will}, in which a generalized linear model based framework is suggested to model the relationship building time. They consider the building time of links as independent random variables coming from a pre-specified distribution and model the expectation as a function of a linear predictor of the extracted topological features. A shortcoming of this model is that we need to exactly specify the underlying distribution of times. We came over this problem by learning the distribution from the data using a non-parametric solution. Furthermore, we considered the temporal dynamics of the network which has been entirely ignored in their work.

\begin{table}[t]
	\scriptsize
	\centering
	\caption{Performance Comparison of Different Methods on Different Datasets}
	\label{table:results2}
	%\tiny
	\begin{tabu} to \columnwidth {c c l X[r] X[r] X[r] X[r] X[r] X[r]}
		\toprule
		%\cmidrule(l){2-3} \cmidrule{5-7}
		Dataset & Feature &
		Model &  MAE &   MRE &   RMSE &   MSLE &   MDAE &  CI \\
		\midrule
		\multirow{8}{*}{\rotatebox{90}{DBLP}}
		& \multirow{4}{*}{\rotatebox{90}{Dynamic}}
		& \npglm  &  $\bm{1.99}$ &  $\bm{0.95}$ &   $\bm{2.43}$ &   $\bm{0.30}$ &  $\bm{1.73}$ & $\bm{0.62}$ \\
		& & \textsc{Wbl-Glm} &  2.33 &  1.10 &   2.85 &   0.36 &   2.08 & 0.58 \\
		& & \textsc{Exp-Glm} &  3.11 &  1.39 &   3.88 &   0.52 &   2.58 & 0.50 \\
		& & \textsc{Ray-Glm} &  4.02 &  1.83 &   4.70 &   0.66 &   3.72 & 0.35 \\
		
		\cmidrule{2-9}                                                                            
		& \multirow{4}{*}{\rotatebox{90}{Static}}                                                  
		& \npglm               &  2.76 &  1.35 &   3.07 &   0.44 &   2.88 & 0.26 \\
		& & \textsc{Wbl-Glm}     &  2.81 &  1.38 &   3.16 &   0.45 &   2.88 & 0.48 \\
		& & \textsc{Exp-Glm}     &  3.28 &  1.57 &   3.70 &   0.53 &   3.30 & 0.14 \\
		& & \textsc{Ray-Glm}     &  5.04 &  2.28 &   5.26 &   0.85 &   5.12 & 0.01 \\
		
		\midrule
		\multirow{8}{*}{\rotatebox{90}{Delicious}}
		& \multirow{4}{*}{\rotatebox{90}{Dynamic}}
		& \npglm  &  $\bm{2.10}$ &  $\bm{1.20}$ &   $\bm{2.55}$ &   $\bm{0.35}$ &   $\bm{2.05}$ & $\bm{0.70}$ \\
		& & \textsc{Wbl-Glm} &  2.37 &  1.31 &   2.89 &   0.40 &   2.16 & 0.57 \\
		& & \textsc{Exp-Glm} &  3.21 &  1.58 &   3.84 &   0.54 &   2.89 & 0.55 \\
		& & \textsc{Ray-Glm} &  3.90 &  2.07 &   4.66 &   0.68 &   3.91 & 0.40 \\
		
		\cmidrule{2-9}                 
		& \multirow{4}{*}{\rotatebox{90}{Static}}                                                  
		& \npglm               &  2.33 &  1.46 &   2.80 &   0.41 &   2.17 & 0.61 \\
		& & \textsc{Wbl-Glm}     &  2.65 &  1.62 &   3.23 &   0.47 &   2.26 & 0.43 \\
		& & \textsc{Exp-Glm}     &  3.35 &  1.91 &   4.17 &   0.59 &   2.75 & 0.35 \\
		& & \textsc{Ray-Glm}     &  4.81 &  2.61 &   5.27 &   0.85 &   4.28 & 0.12 \\
		
		\midrule
		\multirow{8}{*}{\rotatebox{90}{MovieLens}}
		& \multirow{4}{*}{\rotatebox{90}{Dynamic}}
		& \npglm  &  $\bm{2.48}$ &  $\bm{3.08}$ &   $\bm{3.04}$ &   $\bm{0.55}$ &  $\bm{2.14}$ & $\bm{0.70}$ \\
		& & \textsc{Wbl-Glm} &  3.06 &  3.61 &   3.79 &   0.65 &   2.60 & 0.56 \\
		& & \textsc{Exp-Glm} &  3.79 &  2.70 &   4.60 &   0.78 &   3.48 & 0.45 \\
		& & \textsc{Ray-Glm} &  4.98 &  3.58 &   5.63 &   1.05 &   4.83 & 0.33 \\
		
		\cmidrule{2-9}                                                                            
		& \multirow{4}{*}{\rotatebox{90}{Static}}                                                  
		& \npglm               &  2.92 &  3.44 &   3.45 &   0.67 &   3.36 & 0.50 \\
		& & \textsc{Wbl-Glm}     &  2.99 &  3.52 &   3.51 &   0.69 &   3.37 & 0.49 \\
		& & \textsc{Exp-Glm}     &  3.42 &  2.89 &   3.86 &   0.78 &   3.82 & 0.49 \\
		& & \textsc{Ray-Glm}     &  5.32 &  4.06 &   5.62 &   1.17 &   5.70 & 0.20 \\
		
		\bottomrule
	\end{tabu}
\end{table}